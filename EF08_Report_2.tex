\documentclass[12pt,a4paper]{article}
\usepackage[margin= 1 in]{geometry}
\usepackage{listing}
\usepackage{graphicx}
\usepackage{svg}
\usepackage{natbib}
\usepackage[toc,page]{appendix}
\usepackage{setspace} \linespread {1}
\setlength{\parskip}{6pt}
\usepackage{array} %To get rid of paragraph indents
\setlength{\parindent}{0pt}
\usepackage{times}% Use times new roman font
\usepackage{amsmath}
\usepackage{booktabs}
%-------------------------------

\begin{document}
\begin{titlepage}
	\begin{center}
	\textbf{Thesis Title}
	\end{center}
\end{titlepage}


\setcounter{page}{2}

\tableofcontents

\pagebreak

\listoffigures

\pagebreak

\listoftables

\pagebreak
%-------------MAIN BODY-----------
\section{Introduction}
-Talk about problems with battery and choice of landing spot
	-Talk about what others have done to assess the issue of short battery life
	-Comment on problems with their designs (high mass, very specific applications in terms of landing spot, extra battery needed, not being able to yaw, etc)
	-Introduce the work by Hao Wang on one DOF quanser
	-Describe how my solution adds something positive to this area of research

p.9 on logbook
Research on perching (see pg 7 on logbook); stanford and vishwa robotics. Talk about Hao Wang's dissertation?
\subsection{Aims and objectives}
-Get the drone to perch and yaw at the same time and lay out tuning guidelines for perching, and future work needed. Is this idea scalable?
\section{Experimental set-up and testing}
\subsection{Materials used}
	Crazyflie 2.0, crazyradio, vm, joystick. Explain rough architecture of drone control (python client and firmware). Talk about software used to compile/edit code?
\subsection{Dynamic model of drone}
-Show my dynamic model of the drone + equations. Show also expected max roll/pitch angles? Describe here the stick lengths that will be used as well.
\subsection{Perching stabilisation/implementation}
Stages of flight (pg 7, 26/6). Show flowchart?
PID control structure for roll ,pitch, yaw
Setting the thrust low enough such that drone doesn't fly away
Attitude vs rate mode
Explain how I first intended to tune the drone - DO THIS VERY BRIEFLY TO EXPLAIN MORE EXTENSIVELY ON THE DISCUSSION (following the usual procedure of tuning inside loop first, and outside loop later, each dof separately), and how that did not work (show results/pictures in the results section or here?: Results could be a plot of the different behaviours on different surfaces and 1 dof test rig results vs with no test rig) 


\subsection{Experimental set-up}
Models of monopod/perching considered? p.8
p.13 test rig models considered/built and why they were used/modified/discarded. Comment on why height adjustment was discarded as well.


\begin{equation} \label{eq1}
[M]\{\ddot{x}\}+[C]\{\dot{x}\}+[K]\{x\} = \{f\}
\end{equation}

\begin{figure}[h!]
\centering
 % \includegraphics[scale=0.5]{Fig1.pdf}
  \caption{Signal processing flowchart}
  \label{fig1}
\end{figure}
Figure \ref{fig1} shows a schematic of the signal processing equipment. As previously \subsection{Test strategy}

\section{Results} \label{results}
Compare stability with and without base thrust and battery life with and without base thrust (what stick length though? Probably just the best stick length)
Show all discussed with 
\section{Discussion}
PID loop tuning: Show effects of not having good yaw control (show images of "unexplained" falling), using the initial test rig, and what effect tuning with regular procedures had (tuning in rate mode only and then attitude vs tuning everything at the same time, tuning 1 dof and the others, etc. Show how this gave largely wrong PID parameters only useful for a few surfaces with movement, and not a stationary point). Regular tuning, with 1 dof at a time was justified by the fact that thrust requests were independently added to each motor from each axis. According to this, no yaw control would be necessary to properly tune the drone, but this proved not to be correct.

Talk about stick length and sensitivity to unbalanced mass.
\section{Conclusions and future work}
"If I had the knowledge I have now, how would I approach this problem?"
"Future stuff to do"
\begin{itemize}

\item effect of yaw
\item effect of test rig
\item   \end{itemize}


\bibliographystyle{unsrt}
\bibliography{references}

\pagebreak
\begin{appendices}
\section{Mode shapes and natural frequencies considered}

\end{appendices}
\end{document}